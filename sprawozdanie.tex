%TODO:
%	-numeracja rozdziałów?
%	-
\documentclass[a4paper,12pt]{report}


% to jest do polskich znakow
\usepackage[polish]{babel}
\usepackage[T1]{polski}
\usepackage[utf8]{inputenc} 

%najpierw sa informacje o dokumencie
\title{Architektura Komputerow - Projekt}
\author{Rafał Pieniążek \and Hubert Zapała}
\date{28 maja}

\begin{document}
\maketitle
\tableofcontents % to jest do spisu treści

	
	\section{Wstep}

	% deskryptory selektory, co to jest , troche o architekturze
	%zadanie
	%TSS
	\subsection{Tryb chroniony}
	Wykonanie jądra systemu operacyjnego wymaga dogłębnego zrozumienia architektury procesorów z rodziny x86. Procesory te posiadają dodatkowy tryb , zwany trybem chronionym. Jest on szczególnie pomocny, podczas przełączania zadań. Podczas, gdy zadanie modyfikuje pewien obszar pamięci, musi być pewne, że wszystkie pozostałe zadania mają zablokowany dostęp do jego pamięci. W trybie chronionym ten problem został rozwiązany. Wszystkie zadania są w nim od siebie odseparowane, ponieważ dostęp do pamięci jest kontrolowany przez procesor. 

	\subsection{Przejście w tryb chroniony}
	Po włączeniu zasilania procesor jest w trybie rzeczywistym. Przejście w tryb chroniony odbywa się programowo, poprzez wykonanie pewnych instrukcji. Najpierw jednak należy zdefiniować strukturę - globalną tablicę deskryptorów - GDT.  Następnie należy ustawić bit PE(zezwolenie na tryb chroniony). Instrukcja LMSW pozwala nam pobrać całe słowo stanu procesor MSW. Po modyfikacji flagi, używamy WMSW, aby zapisać zmodyfikowane dane.
	\section{Jądro systemu w trybie chronionym}
	
	Podstawową funkcją jądra systemu operacyjnego jest możliwość przełączania zadań. Zadanie, jest to wykonywany program, lub niezależny jego fragment. Podczas przełączania zadań wykorzystywane są :
	\begin{itemize}
	\item{segment stanu zadania - TSS}
	\item{deskryptor segmentu stanu zadania}
	\item{rejestr zadania - TR}
	\item{deskryptor furtki zadania}
	\end{itemize}
Segment stanu zadania(\textit{z ang. Task State Segment}) jest to rekord, wchodzący w skład segmentu danych. 
	\subsection{}
	\section{Zakonczenie}

	

\end{document}


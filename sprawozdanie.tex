%TODO:
%	-
%	-
\documentclass[a4paper,12pt]{article}


% to jest do polskich znakow
\usepackage[polish]{babel}
\usepackage[T1]{polski}
\usepackage[utf8]{inputenc} 
\usepackage{indentfirst}

\textheight=24.5cm
\textwidth=17cm
\topskip=5mm
\topmargin=-15mm
\leftmargin=-1mm
\oddsidemargin=-1mm      %10mm
\evensidemargin=-1mm     %10mm
\renewcommand{\baselinestretch}{1}

%najpierw sa informacje o dokumencie
%\title{Architektura Komputerow - Projekt}
%\author{Rafał Pieniążek \and Hubert Zapała}
%\date{28 maja}

\begin{document}
\thispagestyle{empty}

\noindent
\unitlength=1mm
\fboxrule=1mm
\setlength{\fboxrule}{3pt}\fbox{\LARGE\sf WT/P/17}\\[+2mm]
\begin{tabular}{ll}
\begin{minipage}[t]{110mm}
{\LARGE\sf 
Rafał Pieniążek\\
Hubert Zapała}
\end{minipage}
 &
\begin{minipage}[t]{55mm}
{\Large\sf 
\begin{tabular}{ll}
Ocena:  & \raisebox{-3mm}{\framebox(20,9)[cc]{}}\\
\end{tabular}
}
\end{minipage}
\end{tabular}

\hfill {\Large\sf Oddano: \raisebox{-3mm}{\framebox(30,9)[cc]{}}}\\[+10mm]

\begin{center}
{\huge\sf Jądro systemu operacyjnego}\\[+10mm]
{\Large\sc 
Architektura Komputerów 2 -- projekt\\
%Architektura komputer\'ow (2) -- projekt\\
2014/15\\[+10mm]}
\end{center}

\noindent
{\sc 
\hspace*{70mm}Prowadz\k{a}cy:\\
\hspace*{70mm}dr inż Tadeusz Tomczak\\
}

%\maketitle
\newpage
\tableofcontents % to jest do spisu treści
\newpage
	
	\section{Wstep}
		\subsection{Założenia projektowe i rzeczywiste osiągnięcia}
		W ramach projektu przygotowaliśmy podstawowe jądro systemu operacyjnego w trybie chronionym procesora. Jądro ma możliwość przełączania dwóch zadań za pomocą przerwań sprzętowych, pochodzących od klawiatury. Kod źródłowy jądra napisany jest w Turbo Assemblerze. Jądro to można uruchamiać w DOSBoxie, tak jak zwykły program, jednakże w ramach projektu zakupiliśmy stary komputer z procesorem Pentium III. Planowaliśmy uruchamiać system z dyskietki, ale nie poradziliśmy sobie z połączeniem jądra i bootloadera. Dlatego też jądro uruchamialiśmy z poziomu FreeDos, bez  wyższych sterowników grafiki. Kolejnymi etapami prac było:
			\begin{itemize}
				\item{instalacja środowiska}
				\item{kod przejścia procesora w tryb chroniony}
				\item{obsługa przerwań i wyjątków}
				\item{wykorzystanie pamieci rozszerzonej } (Aktywacja linii A20)
				\item{przełączanie zadań przez furtkę zadania }	
				\item{przełączanie zadań przez przerwania sprzętowe }	
			\end{itemize}
	\subsection{Wyjaśnienie pojęć}
		\begin{enumerate}
				\item{\textbf{segmentacja pamięci w trybie rzeczywistym} - adresy logiczne zawierają dwie składowe(adres bazowy segmentu oraz przesunięcie, czyli offset), które są tłumaczone na jednowymiarową pamięć fizyczną. W architekturze x86 wykorzystywane są 4 segmenty:}
					\begin{itemize}
					\item{programu(rejestr CS)}
					\item{stosu(rejestr(SS)}
					\item{danych(rejestr DS,ES)}
					\end{itemize}
				\item{\textbf{segmentacja pamięci w trybie chronionym} - w trybie tym istotne są rejestry sytemowe zawierające adresy bazowe tablic systemowych, służących do organizacji segmentacji. Rejestry te to:}
					\begin{itemize}
						\item{GDTR(Global Descriptor Table Register) - zawiera liniowy adres bazowy, oraz rozmiar globalnej tablicy deskryptorów}
						\item{IDTR(Interrupt Descriptor Table Register) - zawiera liniowy adres bazowy, oraz rozmiar tablicy deskryptorów przerwań}
						\item{LDTR(Local Descriptor Table Register) - zawiera selektor segmentu tablicy deskryptorów lokalnych}
						\item{TR(Task Register) - zawiera selektor segmentu stanu zadania}
					\end{itemize}
				\item {\textbf{deskryptor,tablice deskryptorów }- 8 bajtowa struktura danych określająca lokalizację segmentu programu w przestrzeni adresowej pamięci, oraz zasady dostępu do pamięci. Deskryptory są przechowywane w pamięci w postaci tablic deskryptorów.}
				\item{\textbf{selektor} jest 16 bitowym rejestrem zapisanym w jednym z rejestrów segmentowych. Jest on pewnego rodzaju wskaźnikiem na deskryptor w tablicy deskryptorów.}
					
				\item{\textbf{przerwanie} jest to zdarzenie, które prowadzi do obslugi przerwania. Źródłami przerwań mogą być podzespoły sprzętowe, wyjątki, lub instrukcje programowe. }
				\item{\textbf{Segment stanu zadania - TSS} (\textit{z ang. Task State Segment}) jest to rekord, wchodzący w skład segmentu danych. Każde zadanie posiada swój TSS, przechowujący bierzący kontekst zadania.}
		\end{enumerate}
	
	
	
	\subsection{Tryb chroniony}
	Procesory z rodziny x86  posiadają dodatkowy tryb , zwany trybem chronionym. Jest on szczególnie pomocny, podczas przełączania zadań. Podczas, gdy zadanie modyfikuje pewien obszar pamięci, musi być pewne, że wszystkie pozostałe zadania mają zablokowany dostęp do jego pamięci. W trybie chronionym ten problem został rozwiązany. Wszystkie zadania są w nim od siebie odseparowane, ponieważ dostęp do pamięci jest kontrolowany przez procesor. 

	\subsection{Przejście w tryb chroniony}
	Po włączeniu zasilania procesor jest w trybie rzeczywistym. Przejście w tryb chroniony odbywa się programowo, poprzez wykonanie pewnych instrukcji. Najpierw jednak należy zdefiniować strukturę - globalną tablicę deskryptorów - GDT.  Następnie należy ustawić bit PE(zezwolenie na tryb chroniony). Instrukcja LMSW pozwala nam pobrać całe słowo stanu procesora -  MSW. Po modyfikacji flagi, używamy WMSW, aby zapisać zmodyfikowane dane.
	
	\section{Jądro systemu w trybie chronionym}
	\subsection{Struktura plików}
		Podczas implementacji projektu podzielilśmy kod programu na następujące pliki:
		\begin{itemize}
			\item{jadro\.asm - plik główny}
			\item{MAKRA.TXT - plik zawierający funkcje m.in. wyświetlania tekstu na ekranie, miękkiego powrotu do RM, inicjowania deskryptorów itd.  }
			\item{PM\_DATA.TXT - określone teksty do wyświetlnia na ekranie}
			\item{PM\_EXC.TXT - procedury obsługi wyjątków}
			\item{PM\_IDT.TXT - poszczególne pozycje tablicy IDT}
			\item{STRUKT.TXT - plik zawierający opisy struktury tj. deskryptor segmentu, furtkę przerwania}
		\end{itemize}
	\subsection{Przełączenie zadań}
			Podstawową funkcją jądra systemu operacyjnego jest możliwość przełączania zadań. Zadanie, jest to wykonywany program, lub niezależny jego fragment. Podczas przełączania zadań wykorzystywane są :
		\begin{itemize}
			\item{segment stanu zadania - TSS}
			\item{deskryptor segmentu stanu zadania}
			\item{rejestr zadania - TR}
			\item{deskryptor furtki zadania}
		\end{itemize}
	\subsection{Praca jądra systemu}
		\subsubsection{Kompilacja i linkowanie}
			Po skompilowaniu programu : \textit{tasm jadro.asm} , a następnie zlinkowaniu go :  \textit{tlink jadro.obj} otrzymujemy plik wykonywalny \textit{jadro.exe}. 
		\subsubsection{Inicjalizacja} 	
			Po  uruchomieniu \textit{jadro.exe} następuje inicjalizacja struktur deskryptorów, po czym procesor przechodzi w tryb chroniony. Na ekranie wyświetlana jest pionowa kreska dzieląca ekran na dwie połowy. Po prawej stronie zostaje uruchamione zadanie główne, jakim jest zegar cyfrowy odliczający czas od 00:00:00.
		\subsubsection{Oprogramowanie kontrolera przerwań}
			Układ obsługi przerwań składa się z dwóch układów 8259A,połączonych kaskadowo.Układ 8259 posiada Jeden z nich pełni funckje master, drugi slave. Dzięki kaskadowemu połączeniu możliwa jest obsługa do 15 przerwań sprzętowych.  
		
		\subsubsection{Obsługa klawiatury}		
			Zewnętrzny kontroler klawiatury rozpoznaje który klawisz został wciśniety (lub zwolniony). Warto zauważyć, że pobrany kod nie jest kodem ASCII danego symbolu. Interesujący jest również fakt,że kod wciśnięcia jest inny niż zwolnienia.Obsługa kontrolera odbywa się poprzez porty buforowe(60h i 61 h) oraz port sterujący (64h). Po wygenerowaniu przerwania kod klawisza odczytywany jest z portu 60h. Po pobraniu wartości trzeba wysłać potwierdzenie odbioru na port 61h. Końcową czynnością jest poinformowanie kontrolera przerwań o zakończeniu obsługi przerwania - co jest wykonywane przez zapis wartości 20h na port 20h.
			\subsubsection{Przełączenie zadania przerwaniem sprzętowym}
			 W architekturze x86 przewidziano możliwość wstawienia furtek zadania bezpośrednio do tablicy deskryptorów przerwań. Wówczas generacja przerwania sprzętowego powoduje przełączenie zadania i wykonanie zadania zagnieżdżonego. Podczas przełączania zadania procesor zapamiętuje kontekst bierzącego zadania w jego segmencie stanu zadania, a następnie odczytuje z TSS kontekst nowego zadania, zawierający selektor segmentu i offset, od którego należy rozpocząć jego realizację.  Następnie wykonywany jest rozkaz skoku odległego \textit{JMP DWORD PTR }.
 	
	\subsubsection{Zadania w projekcie}
		W naszym projekcie zostały zaimplementowane dwa zadania. Jak już zostało powiedziane, jednym z nich jest  zegar odliczający czas od 00:00:00. Drugie zadanie, działające po lewej stronie ekranu to wypisywanie znaku gwiazdki, wraz ze zmieniającym się kolorem tła i czczionki. Przełączenie na to zadanie następuje po wciśnieciu klawisza 1.Aby przerwać wyświetlanie gwiazdek i wznowić pracę zegara należy nacisnąć przycisk 2 na klawiaturze.
	
	
	\section{Zakonczenie}
		Realizacja implementacji prostego jądra systemu operacyjnego okazała się zadaniem niełatwym, lecz możliwym do wykonania. Zagłębiając się w teorię poznawaliśmy coraz więcej skomplikowanych zagadnien i  terminów.  Praktyczna realizacja pozwoliła na dogłebne zrozumienie zasad i metod stosowanych podczas implementacji. 
	

\begin{thebibliography}{999}
\bibitem{aa1} W. Stanisławski, D.Raczyński 
{\em Programowanie systemowe mikroprocesorów rodziny x86},
PWN, Warszawa,2010.

\bibitem{aa2} G.Syck,
{\em Turbo Assembler - Biblia użytkownika}, 
LT\&P, Warszawa, 1994.

%\bibitem{aa3} A. Autor1, B. C. Autor2, 
%{\em Tytu\l{} ksi\k{a}\.zki}, 
%Nazwa Wydawcy, Miejsce Wydania, 1992.


\end{thebibliography}	

\end{document}


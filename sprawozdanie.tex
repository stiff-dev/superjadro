%TODO:
%	-
%	-
\documentclass[a4paper,12pt]{article}


% to jest do polskich znakow
\usepackage[polish]{babel}
\usepackage[T1]{polski}
\usepackage[utf8]{inputenc} 

%najpierw sa informacje o dokumencie
\title{Architektura Komputerow - Projekt}
\author{Rafał Pieniążek \and Hubert Zapała}
\date{28 maja}

\begin{document}
\maketitle
\newpage
\tableofcontents % to jest do spisu treści
\newpage
	
	\section{Wstep}
		\subsection{Założenia projektowe i rzeczywiste osiągnięcia}
		W ramach projektu przygotowaliśmy podstawowe jądro systemu operacyjnego w trybie chronionym procesora. Jądro ma możliwość przełączania dwóch zadań za pomocą przerwań sprzętowych, pochodzących od klawiatury. Kod źródłowy jądra napisany jest w Turbo Assemblerze. Jądro to można uruchamiać w DOSBoxie, tak jak zwykły program, jednakże w ramach projektu zakupiliśmy stary komputer z procesorem Pentium III. Planowaliśmy uruchamiać system z dyskietki, ale nie poradziliśmy sobie z połączeniem jądra i bootloadera. Dlatego też jądro uruchamialiśmy z poziomu FreeDosa, bez  wyższych sterowników grafiki. Kolejnymi etapami prac było:
			\begin{itemize}
				\item{instalacja środowiska}
				\item{kod przejścia procesora w tryb chroniony}
				\item{obsługa przerwań i wyjątków}
				\item{wykorzystanie pamieci rozszerzonej } (Aktywacja linii A20)
				\item{przełączanie zadań przez furtkę zadania }	
				\item{przełączanie zadań przez przerwania sprzętowe }	
			\end{itemize}
	\subsection{Wyjaśnienie pojęć}
		\begin{itemize}
				\item{deskryptor,tablice deskryptorów}
				\item{selektor}
				\item{przerwanie}
				%TUTAJ TRZEBA DOPISAĆ KILKA RZECZY, MOZE COS O ARCHITEKTURZE?
				%zadanie
				%TSS
			\end{itemize}
	
	
	
	\subsection{Tryb chroniony}
	Procesory z rodziny x86  posiadają dodatkowy tryb , zwany trybem chronionym. Jest on szczególnie pomocny, podczas przełączania zadań. Podczas, gdy zadanie modyfikuje pewien obszar pamięci, musi być pewne, że wszystkie pozostałe zadania mają zablokowany dostęp do jego pamięci. W trybie chronionym ten problem został rozwiązany. Wszystkie zadania są w nim od siebie odseparowane, ponieważ dostęp do pamięci jest kontrolowany przez procesor. 

	\subsection{Przejście w tryb chroniony}
	Po włączeniu zasilania procesor jest w trybie rzeczywistym. Przejście w tryb chroniony odbywa się programowo, poprzez wykonanie pewnych instrukcji. Najpierw jednak należy zdefiniować strukturę - globalną tablicę deskryptorów - GDT.  Następnie należy ustawić bit PE(zezwolenie na tryb chroniony). Instrukcja LMSW pozwala nam pobrać całe słowo stanu procesora -  MSW. Po modyfikacji flagi, używamy WMSW, aby zapisać zmodyfikowane dane.
	
	\section{Jądro systemu w trybie chronionym}
	
	Podstawową funkcją jądra systemu operacyjnego jest możliwość przełączania zadań. Zadanie, jest to wykonywany program, lub niezależny jego fragment. Podczas przełączania zadań wykorzystywane są :
	\begin{itemize}
	\item{segment stanu zadania - TSS}
	\item{deskryptor segmentu stanu zadania}
	\item{rejestr zadania - TR}
	\item{deskryptor furtki zadania}
	\end{itemize}
Segment stanu zadania(\textit{z ang. Task State Segment}) jest to rekord, wchodzący w skład segmentu danych. 
%	\subsection{}
	\section{Zakonczenie}

	

\end{document}

